\section*{\de{Projektbeschreibung}\en{Project description}}
%Machen Sie darin den Kontext der Arbeit deutlich (ca. 1 Seite),
%worin die wissenschaftliche Fragestellung besteht (ca. 1 Seite)
%und auf welche bereits erworbenen Kenntnisse Sie aufsetzen wollen.
%Es muss klar werden, was genau Sie tun wollen.
%Insgesamt hat die Beschreibung meist einen Umfang von drei bis fünf Seiten

Where there are students, there is learning.
Where there is learning, there are teachers.
Where there are teachers, there are didactics.
The \enquote{art of learning} evolved over the decades and todays modern,
interconnected world allows wholly new teaching techniques which where not possible before.
Especially the field of computer science is rapidly changing and growing and lectures
which are not teaching basics but delve into more concret technologies need to reevalutate
themselves yearly and change the topics and didatics of the lecture according to new standards,
technologies and methods.

Currently, the lectures \enquote{Netze}, 2nd term BI, and \enquote{Internetkommunikaton}
are teached via theoretical exercises and practical ones.
The theoretical are solveable with pen and paper – and
for the kind of topics they are teaching this is totally fine.
The practical ones, however, are more hands on and need at least a functional computer with network access.
The later ones are more elaborate and require specialed network setups for allowing
the students to experience certrain networkg scenarios for themselves. 
Historically, this special setups are contstructed painstainly via manual 
connection of actual physical devices via physical switches, routers etc. in the net lab.
Not only is this tedious and error-prone for one exercise, 
the configuration must be changed daily for the different student groups.
Furthermore, often one error and one misstep of one student causes,
due to the interconnected nature of the whole experience, a whole network collapse
and nobody is able to continue to work.
The coupling between the exercise and the actual physical stack
is becoming more and more of a problem over the years.

While such a setup does has merits,
for one it teaches hands-on just how fickle a network setup can be,
it does has the mentioned major drawbacks,
which are certainly not the only one.
A few years ago though,
there was no alternative to constructing networks.
Sure, virtualisation was available even then but the effort required for setting
up VMs was not smaller then setting up actual devices.
Last but not least, not everything could be modeled easily via VMs too.
With the rise of such technologies like docker and vagrant this,
however, changed.
Docker allows one of to easily use the native Linux namespaces to create isolated containers and vagrant enables a easily reproducible,
deterministic setup of multiple VMs and interconnections between them.
Furthermore, newest web technologies allow such products like katakode,
which brings a complete shell into the browser,
alongside the necessary tools and programs for using vagrant,
docker and more, basically a complete busybox on steroids.
To put it in a nutshell,
with modern technologies it is entirely possible to take the current physical network
setup of the practical exercises and virtualise it.
This would have, compared to the current setup, several advantages:

(i) The setup would be easily deployable and reproducible which would
allow quick and easy changes in the configuration too.
(ii) The whole infrastructure would be not tied to the physical net-lab
space anymore which results in
(iii) students being able to solve the exercises on their own laptops and wherever they want.
Of course (iv) it would be more error-prone and a student‘s mistake
during the exercise would not affect others.
Nothing comes free though and there are some disadvantages:
Not actually physically connecting the networks and experiencing physical network
failures would remove an important experience for beginners in computer networks.
The actual handling of switches, cables, routers and more is a valuable experience
and the actual failure of devices and troubleshooting of such infrastructure must be taught too.
To make a long story short, it is not desireable to \enquote{modernise}
all and every exercise nor is it necessary.
However, the more specialised scenarios,
which are typically part of higher term lectures and later exercise,
do indeed offer themselves up to be recreated with modern tools.

The goal of the master project is therefore to chose two practical exercises from the lectures
\enquote{Netzte}, 2nd term BI, \enquote{Internetkommunikation}, 4nd term BI or
\enquote{Servicemanagment in Netzen}, all term MI,
and re-implement those with modern technologies – containers, virtualization, web etc.– as appropiate.

Creating exercises, tutorials, anything which teaches something to someone, is always a twofold process: On the one hand there is the actual creation, the typical software engineering process: planning, design, requirements analysis, implementation etc. On the other hand there is the customer but different from a typical customer which is the expert for his needs, our customer is a learner, a student who does not knwo exactly what is the best way to learn this or that topic. Does this mean that we can invent an exercise in isolation, without consulting the target audience? Of course not, however, we need to change the approach and the type of question we ask and feedback we collect.
Unfortunately, the reality of a master project is a somewhat limited time table. Therefore, the focus of the project will be in an actual implemantation as an proof-of-concept of the ways oldfashioned exercises can be modernized.
A complete user evaluation study would better fit in e.g. a master thesis.
Nonetheless, basic questioning and a feedback round for the prototypes in the end would be favorable.

The project will start with a quick research and overview task about the landscape of modern learning didatics, focusing not only on published papers but also on modern learning platforms and tutorials 
%<INSER KATAKODA AND K8S LEARNING AND CODEKATA AND SHIT LIKE THAT LINKS>
(e.g. )
and how these are achieving fun didatics for complex topics.
Simultanously, suitable exercises for the modernizaton from the afromentioned lectures will be chosen in conjuktion with Mr Leischner and Ms Kannen. Criteria for choosing an appropiate exercercise are e.g. how much preparation the net lab need for the exercise, how specialized it is, how basic the topic is, how much fun the current state of the exercise is for the students, among others.
After the exercises are chosen and the research is finished, a first Minimum Viable Product (MVP) of an exercise will be built with a suitable technology. This MVP will iterate on the feedback of Mr Leischner and Ms Kannen, preferable also from students, although this will likely not fit into the time frame.
After a suitable MVP ermerged from the iteration, the MVP will be continued until it reaches protoype status alongside the second chosen exercise. Both of the prototypes are the final product of the master project.

Gleb Ostrowski has experience working as a tutor, he teached freshman and second years during his bachelor times about object-oritientd programming in Java. He also has experience in building virtualized infrastructure, e.g. he operates his own server which is a collection of differennt services in indepedent containers, realized via docker and docker-compose. He also accomulated needed knowledge during the lecture „Servicemanagment in Netzen“ which he sucessfully attented during the winter term 2019/2020.
Gleb Ostrowski is therefore suited on both fronts, the technical side and the didatic one, to create exercises for students. Furthermore, as still a student himself, he is aware of the actual needs and wants of his collegaues which should allow him to prepare an exercise which satisfy the students thirst for modern learning methods.
